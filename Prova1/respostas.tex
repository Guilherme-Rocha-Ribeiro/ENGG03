\documentclass{article}
\usepackage[utf8]{inputenc}
\usepackage[T1]{fontenc}
\usepackage[brazil]{babel}
\usepackage{geometry}
\usepackage{amsmath}
\usepackage{ragged2e}

% Hyperref deve ser um dos últimos pacotes carregados
\usepackage[colorlinks=true, 
            linkcolor=blue, 
            urlcolor=blue, 
            citecolor=blue]{hyperref}

\geometry{a4paper, left=3cm, right=2cm, top=3cm, bottom=2cm}
\setlength{\parindent}{1.25cm}


\begin{document}

\begin{center}
\large\textbf{Avaliação 1 - Métodos Numéricos} \\
\large\textbf{ Universidade Federal da Bahia - 2025.1} \\
\end{center}

\begin{flushleft}
Aluno: Guilherme Rocha Ribeiro \\
Professor: Reiner Requião \\
\end{flushleft}

\section*{Questão 1: Localização de ponto crítico em uma ponte}
\justifying

\subsection*{Análise do contexto e Resultados}
"A função da longarina é receber o carregamento do tabuleiro e das transversinas para distribuí-los a infraestrutura da ponte, ou seja, para os pilares 
que por sua vez distribuem para a fundação. Os principais esforços que atuam na longarina é o esforço de flexão e o esforço de cisalhamento oriundos dos 
momentos fletores e forças cortantes gerados pela solicitação de carregamento"\cite{oliveira2016}(Capítulo 3). Nesse contexto, os esforços sofridos pela longarina são determinantes para a estabilidade estrutural e dispersão de cargas entre as meso e infraestrutura.
Assim, o valor encontrado em $x \approx 1.6057m$ na função $V(x) = 5x^3 - 12x^2 + 7x - 1$ representa o ponto onde a força cortante tende a zero, Essa posição é ideal para o posicionamento de um pilar, pois:

\begin{itemize}
\item Distribuição eficiente de cargas superestrutura transmite as cargas à mesoestrutura, que as conduz à infraestrutura. Um pilar nessa posição otimiza essa transmissão.
\item Minimiza tensões na longarina, evitando falhas estruturais. A força cortante nula indica que a longarina não está sujeita a esforços excessivos nesse ponto.
\end{itemize}


\begin{quote}
"Mesoestrutura: São os pilares e elementos de apoio, tem como função receber as cargas da superestrutura e transmiti-las para a infraestrutura." ( OLIVEIRA, A, M. ; PIEROTT, R, M - Capítulo 4) \cite{oliveira2016}   
\end{quote}

\begin{quote}
"Superestrutura: É a parte útil da obra, por onde se trafega, constitui as vigas e lajes, responsável por receber as cargas da utilização e transmiti-las à meso e infraestrutura." ( OLIVEIRA, A, M. ; PIEROTT, R, M - Capítulo 1) \cite{oliveira2016}
\end{quote}

\subsection*{Análise do Método da Bisseção vs Newton-Raphson}
\begin{itemize}
\item $x \approx 1,\!6057$ (4 casas decimais) em 14 iterações 
\item Comparação com Newton-Raphson: convergência em apenas 3 iterações com o chute inicial em $x_0 = 2.0$.
\end{itemize}






\section*{Questão 2: Vazão em sistema de tubulações}
\subsection*{Equação linear e Resultados do método de Gauss}
\begin{align*}
3q_1 + q_2 + 2q_3 &= 18 \\
2q_1 - q_2 + q_3 &= 8 \\
q_1 + 4q_2 - q_3 &= 10
\end{align*}

\begin{itemize}
\item q1 = 4.0
\item q2 = 2.0
\item q3 = 2.0
\end{itemize}


\subsection*{Coerência Física}
Como o sistema traz uma equação de conservação de massa e energia, os valores do resultado dessa matriz serem positivos e reais é um indicativo de que o sistema é fisicamente coerente.

\subsection*{Método de Eliminação de Gauss}
É um método direto de resolução de sistemas lineares, o qual gera a matriz aumentada do sistema e a coloca na forma de uma matriz triangular superior. Devido ao pivoteamento feito na sessão do código "verificação pivo != 0", essa implementação é mais estável numéricamente do que a apresentada no vídeo \cite{Gauss}, a qual não realiza 
pivoteamento parcial, assim garantindo que o algoritmo não falhe em casos de matrizes que tenham solução, mas que apresentam o elemento matriz[0][0] = 0.

\subsection*{Complexidade}
\begin{itemize}
\item Complexidade: $\mathcal{O}(n^3)$ para matrizes $n \times n$
\item Vantagem: Pivoteamento parcial evita falhas quando $a_{11} = 0$
\item Estabilidade numérica superior à implementação sem pivoteamento
\end{itemize}


\section*{Questão 3: Sistema não lineare em reações industriais}
\begin{equation}\label{eq:sistema}
\begin{cases}
xy + y^2 - 10 = 0 \\
x^2 - \ln(y+1) - 2 = 0 \\
\end{cases}
\end{equation}

\subsection*{Resultados do método de Newton-Raphson}



\begin{itemize}
\item Aproximação inicial: $x_0 = [1.0, 2.0]$
\item Solução $X =[1.7946, 2.3898]$
\end{itemize}
\subsection*{Sensibilidade ao ponto inicial e aplicabilidade}
Como o método de Newton-Raphson é iterativo, a escolha do ponto inicial influencia diretamente na convergência e na velocidade de convergência. Visto que, pontos mais próximos da raiz tendem a convergir mais rapidamente, 
enquanto pontos mais distantes podem levar à divergência do método.
Além disso, também é necessário lidar com os domínios das funções separadamente, como no caso da função \begin{equation*}\ln(y+1) > 0 \implies y > -1\end{equation*} 
Assim, evitando que a divergência ocorra devido a valores indefinidos da função e não por falha do método em si. Entretanto, a sua aplicabilidade não é prejudicada por essa sensibilidade, porque os sistemas de equações não lineares na engenharia 
frequentemente derivam de modelos físicos ou matemáticos com base em fenômenos reais. Assim, é possível fazer uma análise prévia do sistema e escolher uma aproximação inicial plausíve para aplicá-lo ao método.


\begin{thebibliography}{9}
\bibitem{oliveira2016} 
OLIVEIRA, A, M. ; PIEROTT, R, M.
\textit{Análise Estrutural de Pontes}. 
UENF, 2016. Disponível em: \url{https://uenf.br/cct/leciv/files/2016/02/Alexandre-Magno-Alves-de-Oliveira-e-Rodrigo-Moulin-Ribeiro-Pierrot.pdf}

\bibitem{Gauss}
\textsc{Numerical Methods}. \textit{Gaussian Elimination In Python}. YouTube, Disponível em: \url{https://www.youtube.com/watch?v=gAmMxdI0EKs}.

\end{thebibliography}

\end{document}